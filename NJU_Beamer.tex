
\documentclass[10pt,aspectratio=169,mathserif]{beamer}
\usepackage{nju}			 %导入 nju 模板宏包
\usepackage[UTF8]{ctex}      %导入 ctex 宏包,添加中文支持
\usepackage{xeCJK}
\usepackage{amsmath,amsfonts,amssymb,bm}   %导入数学公式所需宏包
\usepackage{color}			 %字体颜色支持
\usepackage{graphicx,hyperref,url}
\usepackage{listings}
\usepackage{booktabs}
\usepackage{multirow}
\usepackage{setspace}

% \setstretch{2}
% \let\oldframetitle\frametitle  
% \renewcommand{\frametitle}[1]{       % Redefine \frametitle
% ​\oldframetitle{#1} \setstretch{2}} 


\beamertemplateballitem		%设置 Beamer 主题
\catcode`\。=\active        %或者=13
\newcommand{。}{.}         %将正文中的“。”号转换为“.”。

\AtBeginSection[]
{
  \begin{frame}
    \frametitle{提纲}
    \tableofcontents[currentsection]
  \end{frame}
}



\title{基于规则的中文分词算法}	        %首页信息设置

\author[高辰潇]{            %个人信息设置
    高辰潇\\[0.3cm]
    181220014@smail.nju.edu.cn\\[0.3cm]
    人工智能学院\\}

% \title{Knowledge Transfer in Reinforcement Learning\footnote{based on \emph{Transfer Learning for Reinforcement Learning Domains: A Survey} and \emph{Transfer in Reinforcement Learning: A Framework and a Survey}}}	        %首页信息设置

% \author[高辰潇]{            %个人信息设置
%     Gao Chenxiao\\[0.3cm]
%     181220014@smail.nju.edu.cn\\[0.3cm]
%     School of AI, Nanjing University}

\date{\today}





\begin{document}

\begin{frame}
\titlepage
\end{frame}

\begin{frame}
\frametitle{提纲}
\tableofcontents
\end{frame}

% beginnning of sections

\section{1. 算法:MMSEG}
\begin{frame}{MMSEG}
  \begin{spacing}{1.5}
  \begin{itemize}
    \item MMSEG\footnote{See http://technology.chtsai.org/mmseg/ for details.}是一种基于规则的中文分词方法。相较前向、后向、双向最大匹配,其本身可以被视作一种引入了消歧规则的前向最大匹配算法。
    \item MMSEG基于训练集词典对句子进行词组划分,在得到长度为3的词组(chunk)后,应用先前制定的规则消解歧义。也正因此,MMSEG也可用于新的消歧规则的性能测试。
    \item 在Simple Setting下,MMSEG算法退化成前向最大匹配;本次实验中我采用的是其Complex Setting版本。
  \end{itemize}
  \end{spacing}
\end{frame}

\begin{frame}{MMSEG}
  \begin{spacing}{1.5}
  \begin{enumerate}[1)]
    \item 在分词时,MMSEG会对当前句子进行切分,遍历并存储所有可能的深度为3的词组切分方式(chunk)。例如“比赛以平局告终”一句,若词典为“[比赛, 平局]”,则MMSEG将存储的切分方式包括:[比, 赛, 以],[比赛, 以, 平]和[比赛, 以, 平局]。
  \end{enumerate}
  \end{spacing}
\end{frame}

\begin{frame}{MMSEG}
  \begin{enumerate}[2)]
    \item 根据以下四种规则,对之前得到的切分方式进行筛选
    \begin{itemize}
      \item \emph{Rule1-Maximum Matching}
        \par 优先选择长度最长的chunk。比如在上面的例子中,将选择[比赛, 以, 平局]。
      \item \emph{Rule2-Maximum Average of Word Lengths}
        \par 优先选择单词平均长度最长的分法。该规则主要针对的是句子长度无法切分出三个单词的情况,比如“国际化”一句,可能的切分方式有[国际化],[国际, 化], [国, 际, 化],将选择[国际化]。
      \item \emph{Rule3-Smallest Variance of Word Length}
        \par 优先选择词的长度方差最小的分法。例如[研究, 生命, 起源]和[研究生, 命, 起源],将选择[研究, 生命, 起源]。
      \item \emph{Rule4-Largest Sum of Degree of Morphemic Freedom of One-character Words}
        \par 优先选择单字自由度之和最大的chunk。单子自由度指的是某个字作为单字出现的频次占其总出现次数的比,衡量一个字本身成为一个单词的可能性。
    \end{itemize}
  \end{enumerate}
\end{frame}

\begin{frame}{MMSEG}
  \begin{spacing}{1.5}
  \begin{enumerate}[3)]
    \item 根据上述规则筛选出唯一的`chunk`三元组后,按照该`chunk`的第一个元素对句子进行切分,然后对剩下的句子重复上一过程直到切分完毕。例如`比赛以平局告终`一句,假设最终选择的chunk为`(2, 1, 2)`(即`[比赛, 以, 平局]`),则分词结果就是`比赛`,然后对剩余的句子`以平局告终`重复上述过程。
    
  \end{enumerate}
  \end{spacing}
\end{frame}

\section{2. 其他处理}

\begin{frame}{其他处理}
  \begin{itemize}
    \item 特殊字串识别
    \begin{itemize}
      \item 标点符号:——,...,---,~~~
      \item email地址、网站地址
      \item 百分数字
      \item 专有名词:IOS12.0
      \item 特殊单位:2020年、5月、31日、9时
    \end{itemize}
    \item 数据集预处理
    \begin{itemize}
      \item 切分风格不一致:[一个] vs [一,个]、[这是] vs [这,是]
    \end{itemize}
  \end{itemize}
\end{frame}

\section{3. 实验结果}
\begin{frame}{实验结果}
  \begin{itemize}
    \item 榜单结果:0.9136020744931637
    \item 无额外数据集结果:0.8858429970997282
  \end{itemize}
\end{frame}

% end of sections
\end{document}